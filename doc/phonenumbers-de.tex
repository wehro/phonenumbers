% phonenumbers package: German manual
% Version 2.5
% Datum: 1. Juli 2022
\documentclass[numbers=noenddot]{scrreprt}
\usepackage[french,ngerman]{babel}
\usepackage{fontspec}
\usepackage[link=off]{phonenumbers}
\usepackage[defernumbers=true]{biblatex}
\usepackage{csquotes}
\usepackage{array}
\usepackage{enumitem}
\usepackage{scrlayer-scrpage}
\usepackage{multicol}
\usepackage{metalogo}
\usepackage[ngerman]{isodate}
\usepackage{cnltx-example}
\usepackage{cnltx-tools}
\usepackage[colorlinks=true,
  allcolors=black,
  bookmarksopen=true,
  bookmarksopenlevel=0,
  bookmarksnumbered=true,
  pdfencoding=auto,
  pdftitle={Setzen von Telefonnummern mit LaTeX},
  pdfsubject={Anleitung zum Paket PHONENUMBERS},
  pdfkeywords={latex phonenumbers Telefonnummern},
  pdfauthor={K. Wehr}]{hyperref}

\setmainfont{TeX Gyre Bonum}
\setmonofont{Latin Modern Mono}[Scale=MatchLowercase]

\setlogokern{La}{-0.25em}
\setlogokern{aT}{-0.05em}
\setlogodrop{0.52ex}

\addtokomafont{disposition}{\rmfamily}
\addtokomafont{descriptionlabel}{\rmfamily}

\setlist[itemize]{itemsep=0.7ex plus0.3ex minus0.2ex}

\setlength{\columnsep}{1,5em}
\setlength{\columnseprule}{0,4pt}

\makeatletter
\renewcommand\@pnumwidth{2em}
\makeatother

\addbibresource{Literatur.bib}

\renewcommand\labelnamepunct{\addcolon\space}

\definecolorscheme{phonecolor}{
  cs => cnltxformalblue,
  option => cnltxbrown,
  cnltx => cnltxgreen
}

\setcnltx{
  add-cmds = {setphonenumbers,phonenumber,href},
  color-scheme = phonecolor,
  add-listings-options = {numbers=none},
  pre-output = {\raggedright}
}

\makeatletter
\setlength{\cnltx@before@skip}{5pt plus1pt minus1pt}
\setlength{\cnltx@after@skip}{1pt plus1pt minus1pt}
\makeatother

\DeclareNewLayer[background,bottommargin,mode=picture,hoffset=7cm,
  contents={\includegraphics{Telefonhoerer}}]{Telefonebene}

\AddLayersToPageStyle{plain}{Telefonebene}
\AddLayersToPageStyle{scrheadings}{Telefonebene}

\ExplSyntaxOn

\NewDocumentEnvironment {Befehlsliste} { }
  {
    \begin{list}{ }
      {
        \setlength{\leftmargin}{0pt}
        \setlength{\itemindent}{-1em}
        \setlength{\parsep}{0pt}
        \setlength{\listparindent}{\parindent}
        \setlength{\itemsep}{\topsep}
      }
  }
  {
    \end{list}
  }

\NewDocumentCommand \Befehlsbeschreibung {m o}
  {
    \item
    \cs{#1}
    \IfValueT{#2}{#2}
    \\
  }

\NewDocumentCommand \Optionsbeschreibung {m m m}
  {
    \item
    \option{#1}~=~\choices{#2}
    \hfill
    Voreinstellung:~\code{#3}
    \\
  }

\ExplSyntaxOff

\newcommand\Abschnittsliteratur[1]{\bgroup
\small
\raggedright
\printbibliography[heading=subbibnumbered,keyword=#1]
\egroup}

\newcommand*\Paket[1]{\textsc{#1}}
\newcommand\phone{\textcolor{cnltx}{\Paket{phone\-numbers}}}

\newcommand\UeberschriftGeltungsbereich{\section{Geltungsbereich}}

\newcommand\UeberschriftAufbau{\section{Aufbau der Nummern}}

\newcommand*\UeberschriftOptionen[1]{\section{Optionen}\label{optionen-#1}}

\newcommand\UeberschriftUngueltig{\section{Ungültige Nummern}}

\newcommand\UeberschriftOrtsvorwahlen{\section{Ortsvorwahlen}}

\newcommand\UeberschriftRegionalvorwahlen{\section{Regionalvorwahlen}}

\newcommand\UeberschriftSondervorwahlen{\section{Sondervorwahlen}}

\newcommand\OptionsvariantenAreaCodeSep{\Optionsbeschreibung{area-code-sep}{slash,brackets,space,hyphen}{slash}}

\newcommand\OptionsbeschreibungAreaCodeSep{Legt fest, wie die Vorwahl von der Teilnehmerrufnummer abgetrennt wird.}

\newcommand\HinweisForeignAreaCodeSep{Für Nummern mit Auslandsvorwahl gibt es die Option
\option{foreign-area-code-sep}
 \sieheAbschnitt{optionen-allgemein}.}

\newcommand\OptionsvariantenAreaCode{\Optionsbeschreibung{area-code}{number,place,place-and-number}{number}}

\newcommand\OptionsbeschreibungAreaCode{Legt fest, wie die Vorwahl dargestellt werden soll: als Nummer, als Ortsname oder als Ortsname mit Nummer.}

\newcommand\WertbeschreibungBrackets{Die Variante \code{brackets} setzt bei Festnetznummern Klammern um die Vorwahl. Bei Mobilfunk- und Sondernummern wird sie wie \code{space} behandelt, da bei solchen Nummern die Vorwahl stets mitgewählt werden muss.}

\newcommand\WertbeschreibungPlaceAndNumber{Die Variante \code{place-and-number} gibt für Nummern ohne Auslandsvorwahl den Ortsnamen bzw. die Bedeutung der Vorwahl zusätzlich zur Vorwahlnummer aus.}

\newcommand\WertbeschreibungPlace{Die Variante \code{place} gibt bei Festnetznummern ohne Auslandsvorwahl den Ortsnamen anstelle der Vorwahlnummer aus. In anderen Fällen bleibt es bei der Ausgabe der Nummer.}

\newcommand\KeineVerlinkung{Nummern ohne Vorwahl können allerdings nicht verlinkt werden
\vglAbschnitt{verlinkung}.}

\newcommand\WarnungWenn{Eine Warnung wird ausgegeben, wenn}

\newcommand*\vglAbschnitt[1]{(\cf\ Abschnitt \ref{#1})}

\newcommand*\sieheAbschnitt[1]{(siehe Abschnitt \ref{#1})}

\newcommand*\vglAnhang[1]{(\cf\ Anhang~\ref{#1})}

\newcommand\Quelle{\noindent\emph{Quelle: }}

\newcommand*\Quellen{\noindent\emph{Quellen: }}

\newcommand\Schmalschrift{\small\fontspec{Libertinus Serif}[Ligatures=TeX]}

\newcommand*\AufbauDEATA[6]{Eine #1 Telefonnummer besteht in der Regel aus einer Vorwahl, die mit einer 0 beginnt, und einer Teilnehmerrufnummer. Die Vorwahl kann zwischen #2 und #3 Stellen haben, die Teilnehmerrufnummer im Festznetz zwischen #4 und #5 Stellen. Zur Trennung von Vorwahl und Teilnehmerrufnummer gibt es unterschiedliche Konventionen
\vglAbschnitt{#6}.
Standardmäßig verwendet das Paket \phone\ einen Schrägstrich.}

\newcommand\AufbauDEATB{Die Ortsvorwahl kann bei Anrufen innerhalb eines Ortsnetzes weggelassen werden.}

\newcommand\AufbauDEATC[1]{\KeineVerlinkung\ Als Alternative empfiehlt sich der Einsatz der Option \option{home-area-code}
\vglAbschnitt{optionen-allgemein},
die in #1 überall verwendet werden kann.\par
Am Ende der Teilnehmerrufnummer kann eine abgetrennte Durchwahlnummer stehen, die durch ein optionales Argument oder einen Bindestrich angegeben wird.}

\newcommand*\AufbauDEATD[1]{Bei Anrufen aus dem Ausland wird die Auslandsvorwahl mit der Landeskennzahl #1 vorangestellt und die führende 0 der Ortsvorwahl weggelassen.}

\newcommand\EMail{wehr}
\newcommand\EMailDomain{abgol}
\newcommand\EMailTopLevelDomain{de}

\begin{document}
\begin{titlepage}
\begin{center}
\includegraphics{Titelbild-de}

\vfill
\large
\textit{Paketautor}

\smallskip
Keno Wehr

\normalsize
\texttt{\EMail@\EMailDomain.\EMailTopLevelDomain}

\bigskip
\large
\textit{Fehlermeldungen}

\smallskip
\normalsize
\url{https://github.com/wehro/phonenumbers/issues}
\end{center}

\bigskip
\noindent Dieses Paket ermöglicht es, Telefonnummern unterschiedlichen nationalen Konventionen entsprechend zu setzen und bei Bedarf auch zu verlinken. Unterstützt werden deutsche, österreichische, französische, britische und nordamerikanische Telefonnummern. Nummern aus anderen Ländern werden rudimentär unterstützt.

\bigskip
\noindent An English version of this manual is available in the file \texttt{phonenumbers-en.pdf}.
\end{titlepage}

\tableofcontents

\addfontfeature{Numbers=Proportional}

\chapter{Schnelleinstieg}
\section{Deutschland}
\begin{sidebyside}
  \phonenumber{0441343396}[83]
\end{sidebyside}
\begin{sidebyside}
  \phonenumber[area-code-sep=brackets]{020432632194}
\end{sidebyside}
\begin{sidebyside}
  \phonenumber[area-code=place,area-code-sep=space]{04066969123}
\end{sidebyside}
\begin{sidebyside}
  \phonenumber[foreign]{0209576342}
\end{sidebyside}

\section{Österreich}
\begin{sidebyside}
  \phonenumber[country=AT]{0176543}[210]
\end{sidebyside}

\begin{sidebyside}
  \phonenumber[country=AT,area-code-sep=brackets]{0225854321}
\end{sidebyside}

\begin{sidebyside}
  \phonenumber[country=AT,area-code=place,area-code-sep=space]{0662654321}
\end{sidebyside}

\begin{sidebyside}
  \phonenumber[country=AT,foreign]{0316456789}
\end{sidebyside}

\section{Frankreich}
\begin{sidebyside}
  \phonenumber[country=FR]{0199006789}
\end{sidebyside}
\begin{sidebyside}
  \phonenumber[country=FR,area-code=place-and-number]{0199006789}
\end{sidebyside}
\begin{sidebyside}
  \phonenumber[country=FR,foreign]{0199006789}
\end{sidebyside}

\section{Großbritannien}
\begin{sidebyside}
  \phonenumber[country=UK]{01514960123}
\end{sidebyside}
\begin{sidebyside}
  \phonenumber[country=UK,area-code-sep=space]{01184960234}
\end{sidebyside}
\begin{sidebyside}
  \phonenumber[country=UK,area-code=place,area-code-sep=space]{02079460345}
\end{sidebyside}
\begin{sidebyside}
  \phonenumber[country=UK,foreign]{02920180456}
\end{sidebyside}

\section{Nordamerika}
\begin{sidebyside}
  \phonenumber[country=US]{2125550123}
\end{sidebyside}
\begin{sidebyside}
  \phonenumber[country=US,area-code-sep=space]{2125550123}
\end{sidebyside}
\begin{sidebyside}
  \phonenumber[country=US,area-code=place-and-number]{2125550123}
\end{sidebyside}
\begin{sidebyside}
  \phonenumber[country=US,trunk-prefix]{2125550123}
\end{sidebyside}
\begin{sidebyside}
  \phonenumber[country=US,foreign]{2125550123}
\end{sidebyside}

\section{Andere Länder}
\begin{sidebyside}
  \phonenumber{+3905523776} % 39 = Italien
\end{sidebyside}

\begin{sidebyside}
  \phonenumber{0035923456789} % 359 = Bulgarien
\end{sidebyside}

\chapter{Allgemeine Prinzipien}
\section{Grundideen des Pakets}
Das Ziel des Pakets \phone\ ist es, das Setzen von Telefonnummern zu ermöglichen, ohne dass der Anwender den internen Aufbau der Nummer oder die nationalen typographischen Konventionen für den Telefonnummernsatz kennen muss.

Beispielsweise werden in den meisten Ländern Vorwahlen graphisch von der Rufnummer getrennt. Für die Darstellung der Telefonnummer mit dem Paket \phone\ braucht der Nutzer nicht zu wissen, welche Ziffern zur Vorwahl und welche zur Rufnummer gehören. Das Paket kennt alle Vorwahlen der unterstützten Länder und stellt diese automatisch richtig dar.
\begin{sidebyside}
  \phonenumber{03023125789}
\end{sidebyside}
Hier wurden die ersten drei Ziffern als Vorwahl von Berlin erkannt. Dagegen können im folgenden Beispiel die ersten sechs Ziffern als Vorwahl von Bad Schandau identifiziert werden.
\begin{sidebyside}
  \phonenumber{03502225789}
\end{sidebyside}

Das Paket hat nicht das Ziel, eine bestimmte Norm umzusetzen; es orientiert sich an den üblichen Gepflogenheiten zur Darstellung von Telefonnummern. Durch das Setzen von Optionen kann der Anwender das Land festlegen, aus dem die Telefonnummer stammt, sowie zwischen mehreren Formatierungsvarianten wählen. Hierzu zählt auch die zusätzliche Ausgabe der Auslandsvorwahl.

Das Paket ist außerdem in der Lage zu überprüfen, ob eine Nummer nach den nationalen Regeln zur Nummernvergabe gültig ist. Ungültige Nummern führen bei der Kompilierung zu Warnmeldungen.

\section{Befehle}
\begin{Befehlsliste}
\Befehlsbeschreibung{usepackage}[\oarg{Optionen}\Marg{phonenumbers}]
Lädt das Paket und stellt die \meta{Optionen} global ein, sodass sie für alle Telefonnummern gelten, für die keine anderen Optionen festgelegt sind. Muss in der Präambel stehen.
\Befehlsbeschreibung{setphonenumbers}[\marg{Optionen}]
Setzt die \meta{Optionen} für alle folgenden Telefonnummern, für die nichts anderes festgelegt ist. Kann in der Präambel oder im Dokumentenkörper verwendet werden.
\Befehlsbeschreibung{phonenumber}[\oarg{Optionen}\marg{Nummer}\oarg{Durchwahl\kern1pt}]
Setzt eine Telefonnummer. Die \meta{Optionen} gelten nur für diese Telefonnummer.

Die \meta{Nummer} kann im lokalen oder im internationalen Format eingegeben werden. Das lokale Format ist nur für unterstützte Länder möglich; wenn das lokale Format verwendet wird, ist das zugehörige Land ist durch die Option
\option{country}
festzulegen
\vglAbschnitt{optionen-allgemein}.
Das internationale Format beginnt stets mit
\code{+}
oder
\code{00}
gefolgt von der Landeskennzahl
\vglAnhang{landeskennzahlen}.

Abgesehen von einem führenden Pluszeichen darf die \meta{Nummer} nur aus Ziffern bestehen. Zur Gliederung können Leerzeichen, Klammern (runde und eckige), Schägstriche sowie Bindestriche eingegeben werden; diese werden ignoriert.

Bei deutschen und österreichischen Telefonnummern kann außerdem eine \meta{Durchwahl\kern1pt} angegeben werden, die an die Nummer angehängt wird. Wenn das optionale Argument fehlt und die \meta{Nummer} einen Bindestrich enthält, wird der Nummernteil nach dem (letzten) Bindestrich als Durchwahl erkannt, sofern nicht nur eine Vorwahl vorangeht.

Nummern aus unterstützten Ländern werden den nationalen Konventionen gemäß gesetzt. Nummern aus nicht unterstützten Ländern werden nach der Auslandsvorwahl in Zweiergruppen gegliedert.
\Befehlsbeschreibung{AreaCodesGeographic}[\oarg{Optionen}]
Gibt eine Liste der Orts- bzw. Regionalvorwahlen für das global eingestellte oder durch eine lokale Option angegebene Land aus.
\Befehlsbeschreibung{AreaCodesNonGeographic}[\oarg{Optionen}]{Gibt eine Liste der Vorwahlen ohne geographischen Bezug (Mobilfunk- und Sondervorwahlen) für das gewählte Land aus.}
\Befehlsbeschreibung{CountryCodes}
Gibt eine Liste der gültigen Landeskennzahlen aus.
\end{Befehlsliste}

\section{Verlinkung von Telefonnummern} \label{verlinkung}
Auf modernen Geräten mit Telefoniefähigkeiten können Links auf Telefonnummern verwendet werden, um die Nummern anzurufen, ohne sie eintippen zu müssen. Das Paket \phone\ erzeugt solche Links, sofern das Paket \Paket{hyperref}
\cite{hyperref}
zusätzlich geladen wird.

Unabhängig davon, wie sie im Text erscheinen, werden die Telefonnummern unter Einschluss der Auslandsvorwahl (beginnend mit +) verlinkt, damit sie von überall aus erreichbar sind.

Das Erscheinungsbild der Links kann über \Paket{hyperref}-Optionen eingestellt werden. Beispielsweise könnte die folgende Zeile in die Präambel eingefügt werden, um alle Links auf Telefonnummern in blauer Farbe darzustellen.

\begin{sourcecode}
  \usepackage[colorlinks=true,urlcolor=blue]{hyperref}
\end{sourcecode}

Auf klassischen Arbeitsplatzrechnern sind solche Links allerdings nicht brauchbar. Für den Fall, dass \Paket{hyperref} benötigt wird, ohne dass Telefonnummern verlinkt werden sollen, steht die Option \option{link}
\sieheAbschnitt{optionen-allgemein} zur Verfügung.

\UeberschriftOptionen{allgemein}
Alle Optionen können global mit Hilfe der Befehle \cs{usepackage} und \cs{setphonenumbers} oder lokal für einzelne Telefonnummern festgelegt werden.
\begin{Befehlsliste}
\Optionsbeschreibung{country}{AT,DE,FR,UK,US}{DE}
Gibt an, aus welchem Land die Telefonnummer stammt.
\begin{center}
\begin{tabular}{>{\ttfamily}ll}
AT & Österreich \\
DE & Deutschland \\
FR & Frankreich \\
UK & Großbritannien \\
US & Vereinigte Staaten, Kanada und weitere Länder
\vglAbschnitt{US-bereich}
\end{tabular}
\end{center}
Bitte kontaktieren Sie den Paketautor, falls Sie Unterstützung für weitere Länder benötigen.

Diese Option wird ignoriert, wenn die Nummer im internationalen Format eingegeben wird. In diesem Fall wird das Land durch die Landeskennzahl bestimmt.

\Optionsbeschreibung{link}{on,off}{on}
Gibt an, ob Telefonnummern verlinkt werden sollen, falls das Paket \Paket{hyperref} geladen ist
\vglAbschnitt{verlinkung}.

\Optionsbeschreibung{foreign}{international,american,european,off}{off}
Gibt an, ob und in welcher Form die Auslandsvorwahl ausgegeben werden soll.

Mit \code{foreign=international} oder einfach \code{foreign} erhält man eine Auslandsvorwahl, die aus einem Pluszeichen gefolgt von der Landeskennzahl (z.\,B. 49 für deutsche Nummern) besteht. In dieser Form kann sie auf Mobiltelefonen direkt verwendet werden. Im Festnetz muss das + durch die am Ort des Anrufers verwendeten internationalen Verkehrsausscheidungsziffern ersetzt werden.
\begin{sidebyside}
  \phonenumber[foreign=international]{03023125789}
\end{sidebyside}
Die Variante \code{american} lässt die Auslandsvorwahl mit 011 beginnen, den in Nordamerika gebräuchlichen internationalen Verkehrsausscheidungsziffern.
\begin{sidebyside}
  \phonenumber[foreign=american]{03023125789}
\end{sidebyside}
Bei Auswahl von \code{european} beginnt die Nummer mit einer Doppelnull, wie sie in den meisten Ländern Europas und weiteren Teilen der Welt verwendet wird.
\begin{sidebyside}
  \phonenumber[foreign=european]{03023125789}
\end{sidebyside}

Bei Nummern aus nicht unterstützten Ländern wird die Wahl
\code{foreign=off}
wie
\code{foreign=\linebreak[0]international} behandelt, \ie\ die Auslandsvorwahl wird bei solchen Nummern immer ausgegeben.

\Optionsbeschreibung{foreign-area-code-sep}{space,brackets}{space}
Legt den Vorwahltrenner für Nummern mit Auslandsvorwahl fest. Die Internationale Fernmeldeunion empfiehlt, in Nummern mit Auslandsvorwahl keine Klammern zu verwenden \cite[3]{ITU-123}.

\begin{sidebyside}
  \setphonenumbers{foreign,foreign-area-code-sep=brackets}
  \phonenumber[country=DE]{0441343396} \\
  \phonenumber[country=UK]{01514960123} \\
  \phonenumber[country=US]{2125550123}
\end{sidebyside}
Die Wahl
\code{brackets}
bleibt bei französischen Nummern ohne Auswirkung, da in Frankreich generell keine Klammern in Telefonnummern verwendet werden.

\Optionsbeschreibung{home-country}{AT,DE,FR,UK,US,none}{none}
Legt das Heimatland fest. Für Nummern aus diesem Land wird unabhängig vom Wert der Option \option{foreign} keine Auslandsvorwahl ausgegeben.
\begin{sidebyside}
  \setphonenumbers{foreign,home-country=FR}
  \phonenumber[country=DE]{0441343396} \\
  \phonenumber[country=FR]{0199006789} \\
  \phonenumber[country=US]{2125550123}
\end{sidebyside}

\Optionsbeschreibung{home-area-code}{\meta{Heimatvorwahl\kern1pt},none}{none}
Legt die Vorwahl Ihres Heimatgebietes fest. Bei Nummern mit dieser Vorwahl wird nur die Teilnehmerrufnummer ausgegeben. Die Verlinkung
\vglAbschnitt{verlinkung}
erfolgt immer einschließlich der Vorwahl.

Vor der Heimatvorwahl sollte das Heimatland mit der Option \option{home-country} festgelegt werden. Erfolgt dies nicht, so wird bei der Festlegung der Heimatvorwahl der aktuelle Wert der Option \option{country} als Heimatland gesetzt.
\begin{sidebyside}
  \setphonenumbers{home-country=US,home-area-code=242,foreign}
  \phonenumber[country=US]{2125550123} \\
  \phonenumber[country=US]{2425550124} \\
  \phonenumber[country=DE]{0258163970}
\end{sidebyside}
Die Angabe einer Heimatvorwahl ist nur für Gebiete zulässig, in denen die Vorwahl bei Ortsgesprächen weggelassen werden kann.
\Optionsbeschreibung{group-min}{3,4,5,6,7}{3}
In mehreren Ländern werden Telefonnummern in Zweiergruppen gegliedert. Der Wert dieser Option gibt an, ab welcher Länge eines Nummernteils (Auslandsvorwahl, Ortsvorwahl, Teilnehmerrufnummer, Durchwahl) eine Gliederung stattfindet.
\begin{sidebyside}
  \setphonenumbers{country=AT}
  \phonenumber{04632503}[364] \\
  \phonenumber[group-min=5]{04632503}[364]
\end{sidebyside}
\end{Befehlsliste}

\UeberschriftUngueltig
Um Robustheit gegenüber Fehlern zu gewährleisten, wird die Kompilierung niemals mit einer Fehlermeldung abgebrochen, wenn dem Befehl \cs{phonenumber} eine ungültige Nummer übergeben wird. Stattdessen schreibt das Paket \phone\ Warnmeldungen in die Log-Datei.
Dies ist der Fall, wenn
\begin{itemize}
\item die Eingabe leer ist oder unerlaubte Zeichen enthält,
\item eine Durchwahlnummer angegeben ist, obwohl es sich nicht um eine deutsche oder österreichische Telefonnummer handelt,
\item eine im internationalen Format eingegebene Nummer nur aus einer Landesvorwahl besteht oder keine gültige Landeskennzahl enthält,
\item die Nummer nicht den nationalen Regeln des gewählten Landes entspricht (nur für unterstützte Länder).
\end{itemize}

\section{Technische Hinweise}
Das Paket \phone\ erfordert das Paket
\Paket{l3keys2e}
.

Bindestriche innerhalb von Telefonnummern werden durch
\verbcode:\kern1pt-\kern1pt: realisiert, das heißt sie werden mit einem Zusatzabstand von 1 Punkt zu den umgebenden Ziffern gesetzt. Das gilt auch für Schrägstriche, die als
\verbcode:\kern1pt\slash\kern1pt: ausgegeben werden, was einen Zeilenumbruch nach dem Schrägstrich ermöglicht. Nach einem Pluszeichen wird ebenfalls ein Zusatzabstand eingefügt
(\verbcode:+\kern1pt:).
Die Gliederung deutscher, österreichischer, französischer, britischer und nicht unterstützer Nummern erfolgt durch kleine Leerzeichen
\verbcode:\,:.

Für die Verlinkung von Telefonnummern wird der \Paket{hyperref}-Befehl \cs{href} verwendet. Sofern \Paket{hyperref} geladen ist, wird der Befehl
\verbcode:\phonenumber{0441654321}:
zu
\begin{center}
\verbcode=\href{tel:+49441654321}{04\,41\kern1pt\slash\kern1pt65\,43\,21}=
\end{center}
expandiert.

\section{Lizenz}
Das Paket \phone\ unterliegt der
\emph{\LaTeX\ Project Public License},
Version 1.3 oder Nachfolgeversion.%
\footnote{\url{http://www.latex-project.org/lppl.txt}}

\Abschnittsliteratur{general}

\chapter{Deutsche Telefonnummern}
\UeberschriftAufbau
\AufbauDEATA{deutsche}{3}{6}{3}{10}{optionen-DE}
\begin{sidebyside}
  \phonenumber{02517654321}
\end{sidebyside}

\AufbauDEATB
\begin{sidebyside}
  \phonenumber{7654321}
\end{sidebyside}
\AufbauDEATC{Deutschland}
\begin{sidebyside}
  \phonenumber{0251123456}[78] \\
  \phonenumber{02286543-210} \\
  \phonenumber{8765}[432] \\
  \phonenumber{964278-53}
\end{sidebyside}

\AufbauDEATD{49}
\begin{sidebyside}
  \phonenumber[foreign]{02517654321}
\end{sidebyside}

In der Regel werden alle Teilnummern von hinten in Zweiergruppen gegliedert. Eine Ausnahme besteht für Vorwahlen, deren letzte Ziffer den Tarif angibt. Dies betrifft sogenannte MABEZ-Nummern (\emph{Massenverkehr zu bestimmten Zielen}, verwendet z.\,B. für Fernsehsendungen mit Zuschauerabstimmung, Vorwahl 01\,37\,X) und Service-Nummern (Vorwahl 01\,80\,X). Hier steht die letzte Ziffer der Vorwahl allein
\cite[110]{duden}.
\begin{sidebyside}
  \phonenumber{01374654832}
\end{sidebyside}
Der zugehörige Tarif kann mit Hilfe der Option
\code{area-code=place-and-number}
\sieheAbschnitt{optionen-DE} ausgegeben werden.

\UeberschriftOptionen{DE}
\begin{Befehlsliste}
\OptionsvariantenAreaCodeSep
\OptionsbeschreibungAreaCodeSep
\begin{sidebyside}
  \phonenumber[area-code-sep=space]{0258163970} \\
  \phonenumber[area-code-sep=hyphen]{01749091317}
\end{sidebyside}

\WertbeschreibungBrackets
\begin{sidebyside}
  \setphonenumbers{area-code-sep=brackets}
  \phonenumber{02581639737} \\
  \phonenumber{01749091317}
\end{sidebyside}

\HinweisForeignAreaCodeSep
\OptionsvariantenAreaCode
\OptionsbeschreibungAreaCode

\WertbeschreibungPlace
\begin{sidebyside}
  \setphonenumbers{area-code=place}
  \phonenumber{08942630845} \\
  \phonenumber{01749091317} \\
  \phonenumber[foreign]{04414363524}
\end{sidebyside}

\WertbeschreibungPlaceAndNumber
\begin{sidebyside}
  \setphonenumbers{area-code=place-and-number}
  \phonenumber{08942630845} \\
  \phonenumber{01749091317} \\
  \phonenumber{01803635341} \\
  \phonenumber[foreign]{04414363524}
\end{sidebyside}
\end{Befehlsliste}

\UeberschriftUngueltig
\WarnungWenn
\begin{itemize}
\item eine mit 0 beginnende Nummer keine gültige Vorwahl enthält,
\item die eingegebene Nummer nur aus einer Vorwahl besteht,
\item eine Mobilfunk- oder Sondernummer eine Durchwahl enthält (außer für MABEZ-Nummern).
\end{itemize}
Bei Festnetznummern wird eine Warnung ausgegeben, wenn
\begin{itemize}
\item die Teilnehmerrufnummer mit einer 0 beginnt \cite[6]{BNA-nummernplan},
\item die Teilnehmerrufnummer einschließlich Durchwahlnummer weniger als 3 Stellen hat \cite[6]{BNA-nummernplan},
\item die Teilnehmerrufnummer einschließlich Durchwahlnummer mehr als 10 Stellen hat \cite[2]{BNA-Struktur},
\item die Nummer einschließlich Vorwahl mehr als 13 Stellen hat \cite[2]{BNA-Struktur}.
\end{itemize}
Bei Mobilfunknummern wird eine Warnung ausgegeben, wenn
\begin{itemize}
\item die Nummer mit 015 beginnt und nicht genau 12 Stellen hat \cite[49\psq]{BNA-konzept},
\item die Nummer mit 016 oder 017 beginnt und weniger als 11 oder mehr als 12 Stellen hat \cite[50]{BNA-konzept},
\end{itemize}
Außerdem wird eine Warnung ausgegeben, wenn
\begin{itemize}
\item eine MABEZ-Nummer (\emph{Massenverkehr zu bestimmten Zielen}, Vorwahl 01\,37\,X) nicht genau 11 Stellen hat \cite[2]{BNA-mabez},
\item eine Funkrufnummer (Nummer eines sogenannten Pagers) mit der Vorwahl 01\,64 mehr als 14 Stellen oder eine Funkrufnummer mit der Vorwahl 01\,68 oder 01\,69 mehr als 15 Stellen hat \cite[2]{BNA-nummernplan},
\item eine Servicenummer (Vorwahl 01\,80\,X) nicht genau 11 Stellen hat \cite[71]{BNA-konzept},
\item eine IVPN-Nummer (\emph{internationales virtuelles privates Netz}, Vorwahl 01\,81) weniger als 8 oder mehr als 15 Stellen hat \cite[55]{BNA-konzept},
\item eine VPN-Nummer (\emph{virtuelles privates Netz}, Vorwahl 01\,8X) nicht genau 12 Stellen hat \cite[53]{BNA-konzept},
\item eine Onlinedienstnummer mit der Vorwahl 01\,91, 01\,92 oder 01\,93 weniger als 5 oder mehr als 14 Stellen hat oder eine Onlinedienstnummer mit der Vorwahl 01\,94 weniger als 7 oder mehr als 14 Stellen hat \cite[\ppno\ 1 u. 4]{BNA-019},
\item eine nationale Teilnehmerrufnummer (Vorwahl 0\,32) nicht genau 12 Stellen hat \cite[45]{BNA-konzept},
\item eine persönliche Rufnummer (Vorwahl 07\,00) nicht genau 12 Stellen hat \cite[74]{BNA-konzept},
\item eine kostenlose Rufnummer (Vorwahl 08\,00) weniger als 11 oder mehr als 14 Stellen hat \cite[\ppno\ 1 u. 5\psq]{BNA-0800},
\item eine Premium-Dienst-Nummer (Vorwahl 0\,90\,0X) nicht genau 11 Stellen hat \cite[76]{BNA-konzept},
\item eine Dialer-Nummer (Vorwahl 0\,90\,09) nicht genau 12 Stellen hat \cite[78]{BNA-konzept}.
\end{itemize}

\Abschnittsliteratur{german}

\chapter{Österreichische Telefonnummern}
\UeberschriftAufbau
\AufbauDEATA{österreichische}{2}{5}{5}{9}{optionen-AT}

Alle Teilnummern werden von hinten in Zweiergruppen gegliedert.
\begin{sidebyside}
  \phonenumber[country=AT]{0225854321}
\end{sidebyside}

\AufbauDEATB
\begin{sidebyside}
  \phonenumber[country=AT]{456789}
\end{sidebyside}
\AufbauDEATC{Österreich}
\begin{sidebyside}
  \setphonenumbers{country=AT}
  \phonenumber{03622345}[67] \\
  \phonenumber{0176543-210} \\
  \phonenumber{8765}[432] \\
  \phonenumber{964278-53}
\end{sidebyside}

\AufbauDEATD{43}
\begin{sidebyside}
  \phonenumber[country=AT,foreign]{0316456789}
\end{sidebyside}

\UeberschriftOptionen{AT}
\begin{Befehlsliste}
\OptionsvariantenAreaCodeSep
\OptionsbeschreibungAreaCodeSep
\begin{sidebyside}
  \setphonenumbers{country=AT}
  \phonenumber[area-code-sep=space]{0225854321} \\
  \phonenumber[area-code-sep=hyphen]{065086754231}
\end{sidebyside}

\WertbeschreibungBrackets
\begin{sidebyside}
  \setphonenumbers{country=AT,area-code-sep=brackets}
  \phonenumber{0225854321} \\
  \phonenumber{065086754231}
\end{sidebyside}

\HinweisForeignAreaCodeSep
\OptionsvariantenAreaCode
\OptionsbeschreibungAreaCode

\WertbeschreibungPlace
\begin{sidebyside}
  \setphonenumbers{country=AT,area-code=place}
  \phonenumber{0316456789} \\
  \phonenumber{065086754231} \\
  \phonenumber[foreign]{0225854321}
\end{sidebyside}

\WertbeschreibungPlaceAndNumber
\begin{sidebyside}
  \setphonenumbers{country=AT,area-code=place-and-number}
  \phonenumber{0316456789} \\
  \phonenumber{065086754231} \\
  \phonenumber[foreign]{0225854321}
\end{sidebyside}
\end{Befehlsliste}

\UeberschriftUngueltig
\WarnungWenn
\begin{itemize}
\item eine mit 0 beginnende Nummer keine gültige Vorwahl enthält,
\item die eingegebene Nummer nur aus einer Vorwahl besteht,
\item eine Mobilfunk- oder Sondernummer eine Durchwahl enthält.
\end{itemize}
Bei Festnetznummern wird eine Warnung ausgegeben, wenn
\begin{itemize}
\item die Teilnehmerrufnummer mit einer 0 oder einer 1 beginnt \cite[§\,50 (9)]{RTR-Verordnung},
\item die Teilnehmerrufnummer weniger als 5 oder mehr als 9 Stellen hat \cite[§\,50 (3) u. (5)]{RTR-Verordnung},
\item eine Teilnehmerrufnummer mit der Vorwahl 01 (Wien) weniger als 7 Stellen hat \cite[§\,50 (4)]{RTR-Verordnung},
\item eine Teilnehmerrufnummer mit der Vorwahl
0\,22\,36 (Mödling), 0\,22\,52 (Baden), 03\,16 (Graz), 04\,63 (Klagenfurt), 05\,12 (Innsbruck), 0\,55\,72 (Dornbirn), 06\,62 (Salzburg), 0\,72\,42 (Wels) oder 07\,32 (Linz)
weniger als 6 Stellen hat \cite[§\,50 (4)]{RTR-Verordnung},
\item die Nummer einschließlich Vorwahl mehr als 13 Stellen hat \cite[§\,50 (5)]{RTR-Verordnung}.
\end{itemize}
Außerdem wird eine Warnung ausgegeben, wenn
\begin{itemize}
\item eine Mobilfunknummer weniger als 11 oder mehr als 13 Stellen hat \cite[§\,61 (1)]{RTR-Verordnung},
\item eine VPN-Nummer (\emph{virtuelles privates Netz}, Vorwahl
05\,0X, 05\,17, 0\,57 oder 0\,59)
weniger als 9 oder mehr als 13 Stellen hat \cite[§\,56 (1) u. §\,4 (4)]{RTR-Verordnung},
\item eine Dial-Up-Nummer (Vorwahl
07\,18 oder 08\,04)
weniger als 10 oder mehr als 13 Ziffern hat \cite[§\,66]{RTR-Verordnung},
\item eine standortunabhängige Nummer (Vorwahl
07\,20)
weniger als 10 oder mehr als 13 Ziffern hat \cite[§\,71]{RTR-Verordnung},
\item eine Nummer für konvergente Dienste (Vorwahl
07\,80)
weniger als 10 oder mehr als 13 Ziffern hat \cite[§\,76]{RTR-Verordnung},
\item eine Nummer für Dienste mit geregelter Entgeltobergrenze (Vorwahl
08\,00, 08\,10, 08\,20, 08\,21 oder 08\,28)
weniger als 9 oder mehr als 13 Ziffern hat \cite[§\,81]{RTR-Verordnung},
\item eine Nummer für Mehrwertdienste (Vorwahl
09\,00, 09\,01, 09\,30, 09\,31 oder 09\,39)
weniger als 10 oder mehr als 13 Ziffern hat \cite[§\,87]{RTR-Verordnung}.
\end{itemize}

\Abschnittsliteratur{austrian}

\chapter{Französische Telefonnummern}
\UeberschriftGeltungsbereich \label{FR-bereich}
Der französische Nummernplan
\cite{ARCEP}
gilt nicht nur für das französische Mutterland, sondern auch für die meisten französischen Überseegebiete. Dies betrifft
\begin{itemize}
\item Guadeloupe (Karibik),
\item Martinique (Karibik),
\item Französisch-Guayana (Südamerika),
\item R\'eunion (Indischer Ozean),
\item Mayotte (Indischer Ozean),
\item Saint-Pierre und Miquelon (vor der Ostküste Kanadas),
\item Saint-Barth\'elemy (Karibik),
\item Saint-Martin (Karibik),
\item die Französischen Süd- und Antarktisgebiete (\emph{\foreignlanguage{french}{Terres australes et antarctiques françaises}}, Indischer Ozean/Antarktis).
\end{itemize}
Er gilt hingegen nicht für die Pazifikgebiete Wallis und Futuna, Französisch-Polynesien und Neukaledonien.

\UeberschriftAufbau
Französische Telefonnummern sind generell zehnstellig und werden in Zweiergruppen gegliedert. Die erste Ziffer ist immer eine 0.
\begin{sidebyside}
  \phonenumber[country=FR]{0199006789}
\end{sidebyside}
Die zweite Ziffer ermöglicht die Zuordnung der Nummer zu einem von fünf geographischen Bereichen bzw. einer besonderen Verwendung (z.\,B. Mobilfunk). Bestimmte Nummerngruppen sind für Überseegebiete reserviert
\vglAnhang{vorwahlen-FR}.
\begin{sidebyside}
  \setphonenumbers{country=FR,area-code=place-and-number}
  \phonenumber{0199006789} \\
  \phonenumber{0536495678} \\
  \phonenumber{0596123456}
\end{sidebyside}

Bei Anrufen aus dem Ausland entfällt die führende 0 der Rufnummer.
\begin{sidebyside}
  \phonenumber[country=FR,foreign]{0199006789}
\end{sidebyside}

Nummern für die maschinelle Kommunikation, die mit 07\,00 beginnen, haben mehr als zehn Ziffern. Sie sind im Mutterland 14-stellig, in den Überseegebieten 13-stellig \cite[18]{ARCEP}.
\begin{sidebyside}
  \phonenumber[country=FR]{07000123456789} \\
  \phonenumber[country=FR]{0700512345678}
\end{sidebyside}

Einige Firmen und Institutionen haben vierstellige Kurznummern, die stets mit einer 3 beginnen.
\begin{sidebyside}
  \phonenumber[country=FR]{3245}
\end{sidebyside}
Kurznummern werden ohne Auslandsvorwahl verlinkt
\vglAbschnitt{verlinkung}, da sie aus dem Ausland nicht erreichbar sind \cite[26]{ARCEP}.

Obwohl die in Abschnitt \ref{FR-bereich} genannten Gebiete intern wie ein einziges Netz behandelt werden, gibt es im Bereich des französichen Nummernplans verschiedene Landeskennzahlen:
\begin{center}
\begin{tabular}{rl}
33 & Mutterland \\
262 & R\'eunion, Mayotte, Französische Süd- und Antarktisgebiete \\
508 & Saint-Pierre und Miquelon \\
590 & Guadeloupe, Saint-Barth\'elemy, Saint-Martin \\
594 & Französisch-Guayana \\
596 & Martinique \\
\end{tabular}
\end{center}
Die erste Nummer im folgenden Beispiel stammt aus Mayotte und verwendet daher die Landeskennzahl 262, die zweite stammt aus dem Mutterland mit der Landeskennzahl 33.
\begin{sidebyside}
  \setphonenumbers{country=FR,foreign}
  \phonenumber{0269123456} \\
  \phonenumber{0261913456}
\end{sidebyside}

Die Auslandsvorwahl führt für einige Gebiete zu einer Verdopplung der ersten drei Ziffern, z.\,B. im Fall von Martinique (Regionalvorwahl 05\,96).
\begin{sidebyside}
  \phonenumber[country=FR,foreign]{0596123456}
\end{sidebyside}
Dagegen entfällt für Saint-Pierre und Miquelon die Regionalvorwahl 05\,08 bei Auslandsanrufen vollständig \cite[13]{ARCEP}.
\begin{sidebyside}
  \phonenumber[country=FR,foreign]{0508123456}
\end{sidebyside}

In Saint-Pierre und Miquelon ist es außerdem möglich, bei lokalen Gesprächen die Vorwahl wegzulassen und nur sechs Ziffern zu wählen \cite[12]{ARCEP}.
\begin{sidebyside}
  \setphonenumbers{country=FR,home-area-code=0508}
  \phonenumber{0508123456}
\end{sidebyside}
Da dies in anderen Gebieten Frankreichs nicht möglich ist, ist \code{0508} der einzige erlaubte Wert für die Option \option{home-area-code} \vglAbschnitt{optionen-allgemein}.

\UeberschriftOptionen{FR}
\begin{Befehlsliste}
\OptionsvariantenAreaCode
\OptionsbeschreibungAreaCode

\WertbeschreibungPlaceAndNumber
\begin{sidebyside}
  \setphonenumbers{country=FR,area-code=place-and-number}
  \phonenumber{0199006789} \\
  \phonenumber{0596123456} \\
  \phonenumber{0612345678} \\
  \phonenumber[foreign]{0199006789}
\end{sidebyside}

\WertbeschreibungPlace
\ Da die Vorwahl in Frankreich stets mitgewählt werden muss, ist von der Verwendung dieser Option abzuraten.
\begin{sidebyside}
  \setphonenumbers{country=FR,area-code=place}
  \phonenumber{0199006789} \\
  \phonenumber{0596123456} \\
  \phonenumber{0612345678} \\
  \phonenumber[foreign]{0199006789}
\end{sidebyside}
\end{Befehlsliste}

\UeberschriftUngueltig
\WarnungWenn
\begin{itemize}
\item die Nummer nicht genau 10 oder 4 Stellen hat (außer bei Nummern, die mit 07\,00 beginnen),
\item eine Nummer mit 10 Stellen nicht mit einer 0 beginnt,
\item eine Nummer mit 4 Stellen nicht mir einer 3 beginnt,
\item eine Nummer mit 10 Stellen keine gültige Vorwahl beinhaltet,
\item eine Nummer, die mit 07\,00 beginnt, nicht 14 oder 13 Stellen hat.
\end{itemize}

\Abschnittsliteratur{french}

\chapter{Britische Telefonnummern}
\nocite{wikipedia-UK}
\nocite{wikipedia-conventions}
\nocite{UK-formatting}
\edef\myindent{\the\parindent}
\begin{minipage}{12cm}\setlength{\parindent}{\myindent}
\UeberschriftGeltungsbereich
Der britische Nummernplan
\cite{Ofcom-plan}
gilt für England, Schottland, Wales, Nordirland, die Insel Man und die Kanalinseln Jersey und Guernsey.

Die folgenden britischen Überseegebiete sind Bestandteil des nordamerikanischen Nummernplans
\sieheAbschnitt{US}: Anguilla, Bermuda, die Britischen Jungferninseln, die Kaiman-Inseln, Montserrat, die Turks- und Caicos-Inseln.

Nummern aus den übrigen Überseegebieten werden nicht unterstützt.
\end{minipage}
\hfill
\adjustimage{valign=c}{Britische_Zelle}

\UeberschriftAufbau
Eine britische Telefonnummer besteht in der Regel aus einer Vorwahl, die mit einer 0 beginnt, und einer Teilnehmerrufnummer. Von wenigen kürzeren Nummern abgesehen sind britische Telefonnummern 11-stellig. Bei Festnetzrufnummern wird die Vorwahl üblicherweise in Klammern gesetzt.
\begin{sidebyside}
  \phonenumber[country=UK]{01514960123}
\end{sidebyside}

In den meisten Ortsnetzen kann die Ortsvorwahl bei lokalen Anrufen weggelassen werden.
\begin{sidebyside}
  \phonenumber[country=UK]{7654321}
\end{sidebyside}
\KeineVerlinkung\ Als Alternative empfiehlt sich die Verwendung der Option \option{home-area-code}
\vglAbschnitt{optionen-allgemein}.

\AufbauDEATD{44}
\begin{sidebyside}
  \phonenumber[country=UK,foreign]{01184960234}
\end{sidebyside}

Vorwahlen mit 6 Ziffern werden durch einen Abstand vor der zweitletzten Ziffer gegliedert, Teilnehmerrufnummern mit 7 oder 8 Ziffern durch einen Abstand vor der viertletzten Ziffer. Kürzere Nummern bleiben ungegliedert.
\begin{sidebyside}
  \setphonenumbers{country=UK}
  \phonenumber{02079460345} \\
  \phonenumber{01697312345} \\
  \phonenumber{07700900123}
\end{sidebyside}

\UeberschriftOptionen{UK}
\begin{Befehlsliste}
\Optionsbeschreibung{area-code-sep}{brackets,space}{brackets}
\OptionsbeschreibungAreaCodeSep
\begin{sidebyside}
  \phonenumber[country=UK,area-code-sep=space]{01514960123}
\end{sidebyside}

\WertbeschreibungBrackets
\begin{sidebyside}
  \setphonenumbers{country=UK,area-code-sep=brackets}
  \phonenumber{02079460345} \\
  \phonenumber{07700900123} \\
  \phonenumber{08081570678}
\end{sidebyside}

\HinweisForeignAreaCodeSep
\OptionsvariantenAreaCode
\OptionsbeschreibungAreaCode

\WertbeschreibungPlace
\begin{sidebyside}
  \setphonenumbers{country=UK,area-code=place}
  \phonenumber{02079460345} \\
  \phonenumber{07700900123} \\
  \phonenumber[foreign]{01184960234}
\end{sidebyside}

\WertbeschreibungPlaceAndNumber
\begin{sidebyside}
  \setphonenumbers{country=UK,area-code=place-and-number}
  \phonenumber{02079460345} \\
  \phonenumber{07700900123} \\
  \phonenumber{08081570678} \\
  \phonenumber[foreign]{01184960234}
\end{sidebyside}
\end{Befehlsliste}

\UeberschriftUngueltig
\WarnungWenn
\begin{itemize}
\item eine Telefonnummer, die mit einer 0 beginnt, keine gültige Vorwahl enthält,
\item eine Teilnehmerrufnummer ohne Vorwahl weniger als 4 oder mehr als 8 Stellen hat,
\item eine Teilnehmerrufnummer ohne Vorwahl mit einer 1 beginnt,
\item eine Telefonnummer nur aus einer Vorwahl besteht,
\item eine Festnetznummer mit der Vorwahl 0169\,77 oder 01XXX weniger als 10 oder mehr als 11 Stellen hat,
\item eine Festnetznummer mit einer anderen Vorwahl nicht genau 11 Stellen hat,
\item eine kostenlose Nummer (Vorwahl 0800) weniger als 10 oder mehr als 11 Stellen hat (Ausnahme: \phonenumber[country=UK]{08001111}),
\item eine Mobilfunk- oder Sondernummer mit einer anderen Vorwahl nicht genau 11 Stellen hat.
\end{itemize}

\Abschnittsliteratur{british}

\chapter{Nordamerikanische Telefonnummern} \label{US}
\nocite{wikipedia-conventions-old}
\UeberschriftGeltungsbereich \label{US-bereich}
Der nordamerikanische Nummernplan
\cite{wikipedia-NANP}
gilt in den Vereinigten Staaten, Kanada, mehreren Karibikstaaten und weiteren Gebieten. Es handelt sich im Einzelnen um
\begin{itemize}
\item Amerikanisch-Samoa (US),
\item Anguilla (GB),
\item Antigua und Barbuda,
\item die Bahamas,
\item Barbados,
\item Bermuda (GB),
\item die Britischen Jungferninseln (\emph{British Virgin Islands}, GB),
\item die Kaiman-Inseln (\emph{Cayman Islands}, GB),
\item Dominica,
\item die Dominikanische Republik,
\item Grenada,
\item Guam (US),
\item Jamaika,
\item Montserrat (GB),
\item die Nördlichen Marianen (\emph{Northern Mariana Islands}, US),
\item Puerto Rico (US),
\item St. Kitts und Nevis,
\item St. Lucia,
\item St. Vincent und die Grenadinen,
\item Sint Maarten (NL)\footnote{Der nördliche Teil der Insel gehört unter dem Namen \emph{Saint-Martin} zum französischen Nummernplan
\sieheAbschnitt{FR-bereich}.},
\item Trinidad und Tobago,
\item die Turks- und Caicos-Inseln (GB),
\item die Amerikanischen Jungferninseln (\emph{United States Virgin Islands}, US).
\end{itemize}

\UeberschriftAufbau
Telefonnummern in den Gebieten des nordamerikanischen Nummernplans sind zehnstellig. Sie bestehen aus einer dreistelligen Regionalvorwahl (\emph{area code}), einer dreistelligen Vermittlungsstellennummer (\emph{central office code}) und einer vierstelligen Teilnehmerrufnummer (\emph{subscriber number}) und werden entsprechend gegliedert.
\begin{sidebyside}
  \phonenumber[country=US]{2125550123}
\end{sidebyside}
Neben der Gliederung durch zwei Bindestriche gibt es noch andere Konventionen
\vglAbschnitt{optionen-US}.

Bei Regionalgesprächen ist es vielerorts möglich, die Vorwahl wegzulassen und nur die letzten sieben Ziffern der Nummer zu wählen.
\begin{sidebyside}
  \phonenumber[country=US]{5550123}
\end{sidebyside}
\KeineVerlinkung\ Alternativ ist in den entsprechenden Gebieten die Verwendung der Option \option{home-area-code} möglich
\vglAbschnitt{optionen-allgemein}. Dies gilt jedoch nicht überall \cite{NANPA-ten-digit}, da manchen Regionen aufgrund von Nummernknappheit mehrere Vorwahlen zugeteilt wurden (sogenannte \emph{overlays}).

Bei Ferngesprächen muss in der Regel die Verkehrsausscheidungsziffer 1 (\emph{trunk prefix}) vorgewählt werden.
\begin{sidebyside}
  \phonenumber[country=US,trunk-prefix]{2125550123}
\end{sidebyside}

Für den Mobilfunk gibt es im nordamerikanischen Nummernplan keine eigenen Vorwahlen. Mobiltelefonnummern erhalten gewöhnliche Regionalvorwahlen.

Alle Gebiete des nordamerikanischen Nummernplans sind aus dem Ausland unter der Vorwahl +\kern1pt1 zu erreichen.
\begin{sidebyside}
  \phonenumber[country=US,foreign]{2125550123}
\end{sidebyside}

\UeberschriftOptionen{US}
\begin{Befehlsliste}
\Optionsbeschreibung{area-code-sep}{brackets,space,hyphen}{hyphen}
Legt fest, wie die Vorwahl von der Vermittlungsstellennummer abgetrennt wird.

Da die Vorwahl in manchen Gebieten entfallen kann, ist es möglich, diese in Klammern zu setzen.
\begin{sidebyside}
  \setphonenumbers{country=US,area-code-sep=brackets}
  \phonenumber{2075550123} \\
  \phonenumber[trunk-prefix]{2075550123}
\end{sidebyside}

In Quebec wird die Vorwahl durch Leerschritte abgetrennt
\cite{wikipedia-conventions}.
\begin{sidebyside}
  \setphonenumbers{country=US,area-code-sep=space}
  \phonenumber{4185550123} \\
  \phonenumber[trunk-prefix]{4185550123} \\
\end{sidebyside}

\HinweisForeignAreaCodeSep
\OptionsvariantenAreaCode
\OptionsbeschreibungAreaCode

Die Variante \code{place-and-number} gibt für Nummern ohne Auslandsvorwahl die Region bzw. die Bedeutung der Vorwahl zusätzlich zur Vorwahlnummer aus.
\begin{sidebyside}
  \setphonenumbers{country=US,area-code=place-and-number}
  \phonenumber{4415550125} \\
  \phonenumber{8005550126} \\
  \phonenumber[trunk-prefix]{2125550123} \\
  \phonenumber[foreign]{2125550123}
\end{sidebyside}

Die Variante \code{place} gibt bei geographischen Nummern ohne Auslandsvorwahl und Verkehrsausscheidungsziffer den Ortsnamen anstelle der Vorwahlnummer aus. In anderen Fällen bleibt es bei der Ausgabe der Nummer.
\begin{sidebyside}
  \setphonenumbers{country=US,area-code=place}
  \phonenumber{2125550123} \\
  \phonenumber{4415550125} \\
  \phonenumber{8005550126} \\
  \phonenumber[trunk-prefix]{2125550123} \\
  \phonenumber[foreign]{2125550123}
\end{sidebyside}
Da aufgrund der Vergabe mehrer Vorwahlen für manche Regionen die Vorwahl nicht sicher aus dem Namen der Region rekonstruiert werden kann, wird die Verwendung der Option \code{area-code=place} nicht empfohlen.

\Optionsbeschreibung{trunk-prefix}{on,off}{off}
Gibt an, ob die Verkehrsausscheidungsziffer 1 für Ferngespräche ausgegeben werden soll. Statt \code{trunk-prefix=on} kann einfach \code{trunk-prefix} angegeben werden.
\begin{sidebyside}
  \setphonenumbers{country=US,trunk-prefix=on}
  \phonenumber{2125550123} \\
  \phonenumber{4415550125} \\
  \phonenumber[trunk-prefix=off]{2125550123} \\
  \phonenumber[foreign]{2125550123}
\end{sidebyside}
\end{Befehlsliste}

\newpage
\UeberschriftUngueltig
\WarnungWenn
\begin{itemize}
\item eine Nummer nicht genau 7 oder 10 Stellen hat,
\item eine 10-stellige Nummer keine gültige Vorwahl enthält,
\item die Vermittlungsstellennummer mit einer 0 oder 1 beginnt,
\item die Vermittlungsstellennummer bei einer regionalen Nummer auf 11 endet,
\item die Vermittlungsstellennummer bei einer Sondernummer 911 lautet.
\end{itemize}

\Abschnittsliteratur{american}

\appendix

\setlength\columnseprule{0pt}

\chapter{Deutsche Vorwahlen}
\UeberschriftOrtsvorwahlen
\begin{multicols}{2}
\Schmalschrift
\AreaCodesGeographic[country=DE]
\end{multicols}
\Quelle \cite{BNA-ortsvorwahlen}

\UeberschriftSondervorwahlen
\begin{multicols}{2}
\Schmalschrift
\AreaCodesNonGeographic[country=DE]
\end{multicols}
\Quellen \cite[3--5]{BNA-nummernplan}, \cite{BNA-mobil}, \cite{BNA-0137}, \cite{BNA-0180}

\chapter{Österreichische Vorwahlen}
\UeberschriftOrtsvorwahlen
\begin{multicols}{2}
\Schmalschrift
\AreaCodesGeographic[country=AT]
\end{multicols}
\Quellen \cite{RTR-Liste}, \cite{wikipedia-AT}

\UeberschriftSondervorwahlen
\begin{multicols}{2}
\Schmalschrift
\AreaCodesNonGeographic[country=AT]
\end{multicols}
\Quellen \cite{RTR-Liste}, \cite{wikipedia-AT}

\chapter{Französische Vorwahlen} \label{vorwahlen-FR}
\UeberschriftRegionalvorwahlen
\bgroup
\Schmalschrift
\AreaCodesGeographic[country=FR]
\egroup
\Quellen \cite[13\psq]{ARCEP}, \cite{wikipedia-FR-fr}, \cite{wikipedia-FR-de}

\UeberschriftSondervorwahlen
\bgroup
\Schmalschrift
\AreaCodesNonGeographic[country=FR]
\egroup
\Quellen \cite[16--24]{ARCEP}, \cite{wikipedia-FR-fr}, \cite{wikipedia-FR-de}

\chapter{Britische Vorwahlen}
\UeberschriftOrtsvorwahlen
\begin{multicols}{2}
\Schmalschrift
\AreaCodesGeographic[country=UK]
\end{multicols}
\Quellen \cite{Ofcom-plan}, \cite{UK-area-codes}

\UeberschriftSondervorwahlen
\begin{multicols}{2}
\Schmalschrift
\AreaCodesNonGeographic[country=UK]
\end{multicols}
\Quellen \cite{Ofcom-plan}, \cite{Ofcom-numbering}, \cite{UK-area-codes}

\chapter{Vorwahlen des nordamerikanischen Nummernplans}
\UeberschriftRegionalvorwahlen
\begin{multicols}{2}
\Schmalschrift
\AreaCodesGeographic[country=US]
\end{multicols}
\Quellen \cite{NANPA-geographic}, \cite{NANPA-ten-years}, \cite{NANPA-not-yet}

\UeberschriftSondervorwahlen
\bgroup
\Schmalschrift
\AreaCodesNonGeographic[country=US]
\egroup
\Quellen \cite{NANPA-non-geographic}, \cite{NANPA-ten-years}

\chapter{Gültige Landeskennzahlen\label{landeskennzahlen}}
\begin{multicols}{7}
\Schmalschrift
\noindent
\CountryCodes
\end{multicols}
\Quelle \cite{ITU-164}

\chapter{Versionsprotokoll}
\small
\begin{description}
\item[1.0] \printdate{22.8.2016}
\item[1.1] \printdate{6.11.2016}
\begin{itemize}
\item Verlinkung von Telefonnummern mit \Paket{hyperref}
\item Einführung der Option \option{home-area-code} für die Heimatvorwahl
\item Ergänzung der Mobilfunkvorwahlen der französischen Überseegebiete
\item Ergänzung der neuen nordamerikanischen Vorwahlen 332, 463, 564, 680, 726, 838 und 986
\end{itemize}
\item[1.1.1] \printdate{13.11.2016}
\begin{itemize}
\item Fehlerkorrektur in der Anleitung bezüglich der Option \option{home-area-code}
\end{itemize}
\item[1.2] \printdate{5.3.2017}
\begin{itemize}
\item Einführung der Option
\option{home-country}
für das Heimatland
\item Verwendung der Option
\option{home-country}
anstelle von
\option{country}
zur Festlegung des Landes der Heimatvorwahl
\item Einführung des Befehls
\cs{CountryCodes}
\item Nummerneingabe im internationalen Format
\item rudimentäre Unterstützung für Telefonnummern aus nicht unterstützten Ländern
\item Zusatzabstand (Kerning) von 1 Punkt vor und nach einem Schrägstrich sowie nach einem Pluszeichen
\item Ergänzung der neuen nordamerikanischen Vorwahlen 223 und 445
\end{itemize}
\item[1.2.1] \printdate{12.3.2017}
\begin{itemize}
\item Erlaubnis von 9-stelligen Teilnehmerrufnummern mit Durchwahl im deutschen Festnetz
\item Fehlerkorrektur im Paketcode
\item Änderung einiger Dateinamen
\end{itemize}
\item[2.0] \printdate{22.10.2017}
\begin{itemize}
\item Unterstützung britischer und österreichischer Telefonnummern
\item Einführung der Option
\option{foreign-area-code-sep}
für den Vorwahltrenner in Nummern mit Auslandsvorwahl
\item Eingabe internationaler Nummern mit
\code{00}
am Anfang
\item Ignorieren von Klammern, Schrägstrichen und Bindestrichen in der Eingabe
\item Erkennung eines Bindestrichs in deutschen und österreichischen Nummern als Durchwahltrenner, wenn ein optionales Argument fehlt
\item Erlaubnis von bis zu 10-stelligen (statt 8-stelligen) Teilnehmerrufnummern und insgesamt bis zu 13-stelligen (statt 12-stelligen) Nummern im deutschen Festnetz
\item Erlaubnis von bis zu 14-stelligen (statt 5- bzw. 7-stelligen) 019er-Nummern in Deutschland
\item Erlaubnis von bis zu 14-stelligen (statt 11-stelligen) 0800er-Nummern in Deutschland
\item Ergänzung der neuen nordamerikanischen Vorwahlen 279, 367, 640, 820, 833 und 879
\item Gliederung nicht unterstützter Nummern in Zweiergruppen
\end{itemize}
\item[2.0.1] \printdate{3.12.2017}
\begin{itemize}
\item leichte Verbesserung der Anleitung
\end{itemize}
\item[2.0.2] \printdate{2.1.2018}
\begin{itemize}
\item Fehlerkorrektur im Paketcode
\end{itemize}
\item[2.1] \printdate{5.8.2018}
\begin{itemize}
\item Erlaubnis von Durchwahlen in deutschen MABEZ-Nummern (Vorwahl~01\,37\,X)
\item Ergänzung der neuen deutschen Mobilfunkvorwahl 01\,55\,66
\item Ergänzung der neuen britischen Mobilfunkvorwahlen 07307, 07308, 07309, 07310, 07311 und 07312
\item Entfernung der britischen Vorwahlen 07007, 07571, 07643 und 07662
\item Ergänzung der neuen nordamerikanischen Vorwahlen 326, 521, 658 und 672
\item Entfernung der nordamerikanischen Vorwahl 456
\end{itemize}
\item[2.2] \printdate{18.8.2019}
\begin{itemize}
\item Ermöglichung von Makros als Befehlsargumenten
\item leichte Veränderung der Warnmeldungen
\end{itemize}
\item[2.3] \printdate{20.8.2021}
\begin{itemize}
\item Ergänzung der neuen deutschen Mobilfunkvorwahl 01\,50\,19
\item Entfernung der deutschen Mobilfunkvorwahlen 01\,50\,20, 01\,50\,50, 01\,50\,80, 01\,55\,55
\item Berücksichtigung der mit 07\,00 beginnenden französischen Machine-to-machine-Nummern (einschließlich derjenigen aus den Überseegebieten) mit 14 bzw. 13 Stellen
\item Berücksichtigung der mit 09\,47 und 09\,76 beginnenden Nummern aus den französischen Überseegebieten
\item Ergänzung der neuen britischen Vorwahlen 05604, 07354, 07355, 07357, 07359, 07360, 07361, 07362, 07363, 07364 und 0897
\item Entfernung der britischen Vorwahl 07439
\item Ergänzung der neuen nordamerikanischen Vorwahlen 227, 274, 341, 368, 428, 447, 448, 464, 474, 523, 524, 525, 526, 572, 582, 656, 659, 689, 730, 742, 771, 826, 839, 840, 943, 945 und 948
\item Revision des Paketcodes
\end{itemize}
\item[2.4] \printdate{13.5.2022}
\begin{itemize}
\item Einführung der Option
\option{group-min}, um kurze Nummern nicht zu gliedern
\end{itemize}
\item[2.5] \today
\begin{itemize}
\item neue Dateistruktur des Paketcodes: eine Moduldatei für jedes Land
\item Entfernung der \Paket{ltxcmds}-Abhängigkeit
\item Ergänzung der neuen deutschen Mobilfunkvorwahlen 01\,51\,80, 01\,51\,81, 01\,53\,10, 01\,55\,10 und 01\,55\,11
\item Ergänzung der neuen britischen Mobilfunkvorwahlen 07356 und 07358
\item Entfernung der britischen Vorwahlen 07020, 07050 und 07092
\item Ergänzung der neuen nordamerikanischen Vorwahlen 235, 263, 283, 324, 327, 329, 350, 354, 363, 382, 468, 472, 527, 528, 557, 584, 645, 683, 728, 753, 835, 861, 975 und 983
\item Entfernung der nordamerikanischen Vorwahl 879
\item Aufteilung des Literaturverzeichnisses
\end{itemize}
\end{description}

\end{document}
